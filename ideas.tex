\documentclass[11pt,a4paper]{article}

\usepackage{microtype}
\usepackage{geometry}
\usepackage{amsmath}
\usepackage{xfrac}

\newcommand{\alphamack}{\alpha _{\text{Mack}}}
\newcommand{\alphahall}{\alpha _{\text{Hall}}}

\renewcommand{\thesubsection}{\arabic{subsection}}
\renewcommand{\thesubsubsection}{\quad{} \alph{subsubsection}}

\begin{document}

\section*{Changes to the Rescorla-Wagner Modified Model}

\subsection{Updating $\alpha$}

We hereby present 3 alternatives.

\subsubsection{Double $\alpha$}
\begin{equation}
	\begin{aligned}
		\alpha ^{n + 1}
			&= \alpha ^n + \Delta \alpha ^n \\
			&= \alpha ^n + \left[ \lambda \alphamack ^n + (1 - \lambda) \alphahall ^n \right] 
	\end{aligned}
\end{equation}

$\alphamack$ reflects the Mackintosh formulation of attention for the Rescorla-Wagner model.
\begin{equation}
	\alphamack ^n = f ( \lambda - V ) \\
\end{equation}

$\alphahall$ reflect the Hall and Pearce phenomenon.
\begin{gather}
	\begin{aligned}
		\alphahall ^n &= - \alpha ^n \cdot \delta \cdot e^{- \frac{1}{2} \cdot 
		\displaystyle{ \left( \nabla_1 [ f ] ( n ) \right) ^2 }} \\
		&= - \alpha ^n \cdot \delta e ^ {- \frac{1}{2} \cdot \displaystyle{\left(V^n_{\text{MA}} - V^{n - 1}_{MA} \right) ^2}}
	\end{aligned} \\[1em]
	\begin{aligned}
		\delta &\in ( 0, 1 ) \\
		V ^n _{\text{MA}} (k) &= \frac{1}{k} \sum ^n _{i = n - k + 1} V ^i
	\end{aligned} \notag{}
\end{gather}

\subsubsection{Maximum of both}

\begin{equation}
	\alpha ^{n + 1} = \max { \left( \alphamack ^n, \alphahall ^n \right) } 
\end{equation}

\subsubsection{Thresholding}

\begin{equation}
	\alpha ^{n + 1} = \begin{cases}
		\alphamack ^n & \text{if } V^n > \tau \\
		\alphahall ^n & \text{otherwise}
	\end{cases}
\end{equation}

\end{document}
